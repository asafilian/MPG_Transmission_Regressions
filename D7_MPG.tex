\documentclass[]{article}
\usepackage{lmodern}
\usepackage{amssymb,amsmath}
\usepackage{ifxetex,ifluatex}
\usepackage{fixltx2e} % provides \textsubscript
\ifnum 0\ifxetex 1\fi\ifluatex 1\fi=0 % if pdftex
  \usepackage[T1]{fontenc}
  \usepackage[utf8]{inputenc}
\else % if luatex or xelatex
  \ifxetex
    \usepackage{mathspec}
  \else
    \usepackage{fontspec}
  \fi
  \defaultfontfeatures{Ligatures=TeX,Scale=MatchLowercase}
\fi
% use upquote if available, for straight quotes in verbatim environments
\IfFileExists{upquote.sty}{\usepackage{upquote}}{}
% use microtype if available
\IfFileExists{microtype.sty}{%
\usepackage{microtype}
\UseMicrotypeSet[protrusion]{basicmath} % disable protrusion for tt fonts
}{}
\usepackage[margin=1in]{geometry}
\usepackage{hyperref}
\PassOptionsToPackage{usenames,dvipsnames}{color} % color is loaded by hyperref
\hypersetup{unicode=true,
            pdftitle={Analysis of Mile per Gallon vs.~Tranmission via Regression Models},
            pdfauthor={Aliakbar Safilian},
            colorlinks=true,
            linkcolor=Maroon,
            citecolor=Blue,
            urlcolor=blue,
            breaklinks=true}
\urlstyle{same}  % don't use monospace font for urls
\usepackage{color}
\usepackage{fancyvrb}
\newcommand{\VerbBar}{|}
\newcommand{\VERB}{\Verb[commandchars=\\\{\}]}
\DefineVerbatimEnvironment{Highlighting}{Verbatim}{commandchars=\\\{\}}
% Add ',fontsize=\small' for more characters per line
\usepackage{framed}
\definecolor{shadecolor}{RGB}{248,248,248}
\newenvironment{Shaded}{\begin{snugshade}}{\end{snugshade}}
\newcommand{\KeywordTok}[1]{\textcolor[rgb]{0.13,0.29,0.53}{\textbf{#1}}}
\newcommand{\DataTypeTok}[1]{\textcolor[rgb]{0.13,0.29,0.53}{#1}}
\newcommand{\DecValTok}[1]{\textcolor[rgb]{0.00,0.00,0.81}{#1}}
\newcommand{\BaseNTok}[1]{\textcolor[rgb]{0.00,0.00,0.81}{#1}}
\newcommand{\FloatTok}[1]{\textcolor[rgb]{0.00,0.00,0.81}{#1}}
\newcommand{\ConstantTok}[1]{\textcolor[rgb]{0.00,0.00,0.00}{#1}}
\newcommand{\CharTok}[1]{\textcolor[rgb]{0.31,0.60,0.02}{#1}}
\newcommand{\SpecialCharTok}[1]{\textcolor[rgb]{0.00,0.00,0.00}{#1}}
\newcommand{\StringTok}[1]{\textcolor[rgb]{0.31,0.60,0.02}{#1}}
\newcommand{\VerbatimStringTok}[1]{\textcolor[rgb]{0.31,0.60,0.02}{#1}}
\newcommand{\SpecialStringTok}[1]{\textcolor[rgb]{0.31,0.60,0.02}{#1}}
\newcommand{\ImportTok}[1]{#1}
\newcommand{\CommentTok}[1]{\textcolor[rgb]{0.56,0.35,0.01}{\textit{#1}}}
\newcommand{\DocumentationTok}[1]{\textcolor[rgb]{0.56,0.35,0.01}{\textbf{\textit{#1}}}}
\newcommand{\AnnotationTok}[1]{\textcolor[rgb]{0.56,0.35,0.01}{\textbf{\textit{#1}}}}
\newcommand{\CommentVarTok}[1]{\textcolor[rgb]{0.56,0.35,0.01}{\textbf{\textit{#1}}}}
\newcommand{\OtherTok}[1]{\textcolor[rgb]{0.56,0.35,0.01}{#1}}
\newcommand{\FunctionTok}[1]{\textcolor[rgb]{0.00,0.00,0.00}{#1}}
\newcommand{\VariableTok}[1]{\textcolor[rgb]{0.00,0.00,0.00}{#1}}
\newcommand{\ControlFlowTok}[1]{\textcolor[rgb]{0.13,0.29,0.53}{\textbf{#1}}}
\newcommand{\OperatorTok}[1]{\textcolor[rgb]{0.81,0.36,0.00}{\textbf{#1}}}
\newcommand{\BuiltInTok}[1]{#1}
\newcommand{\ExtensionTok}[1]{#1}
\newcommand{\PreprocessorTok}[1]{\textcolor[rgb]{0.56,0.35,0.01}{\textit{#1}}}
\newcommand{\AttributeTok}[1]{\textcolor[rgb]{0.77,0.63,0.00}{#1}}
\newcommand{\RegionMarkerTok}[1]{#1}
\newcommand{\InformationTok}[1]{\textcolor[rgb]{0.56,0.35,0.01}{\textbf{\textit{#1}}}}
\newcommand{\WarningTok}[1]{\textcolor[rgb]{0.56,0.35,0.01}{\textbf{\textit{#1}}}}
\newcommand{\AlertTok}[1]{\textcolor[rgb]{0.94,0.16,0.16}{#1}}
\newcommand{\ErrorTok}[1]{\textcolor[rgb]{0.64,0.00,0.00}{\textbf{#1}}}
\newcommand{\NormalTok}[1]{#1}
\usepackage{longtable,booktabs}
\usepackage{graphicx,grffile}
\makeatletter
\def\maxwidth{\ifdim\Gin@nat@width>\linewidth\linewidth\else\Gin@nat@width\fi}
\def\maxheight{\ifdim\Gin@nat@height>\textheight\textheight\else\Gin@nat@height\fi}
\makeatother
% Scale images if necessary, so that they will not overflow the page
% margins by default, and it is still possible to overwrite the defaults
% using explicit options in \includegraphics[width, height, ...]{}
\setkeys{Gin}{width=\maxwidth,height=\maxheight,keepaspectratio}
\IfFileExists{parskip.sty}{%
\usepackage{parskip}
}{% else
\setlength{\parindent}{0pt}
\setlength{\parskip}{6pt plus 2pt minus 1pt}
}
\setlength{\emergencystretch}{3em}  % prevent overfull lines
\providecommand{\tightlist}{%
  \setlength{\itemsep}{0pt}\setlength{\parskip}{0pt}}
\setcounter{secnumdepth}{5}
% Redefines (sub)paragraphs to behave more like sections
\ifx\paragraph\undefined\else
\let\oldparagraph\paragraph
\renewcommand{\paragraph}[1]{\oldparagraph{#1}\mbox{}}
\fi
\ifx\subparagraph\undefined\else
\let\oldsubparagraph\subparagraph
\renewcommand{\subparagraph}[1]{\oldsubparagraph{#1}\mbox{}}
\fi

%%% Use protect on footnotes to avoid problems with footnotes in titles
\let\rmarkdownfootnote\footnote%
\def\footnote{\protect\rmarkdownfootnote}

%%% Change title format to be more compact
\usepackage{titling}

% Create subtitle command for use in maketitle
\newcommand{\subtitle}[1]{
  \posttitle{
    \begin{center}\large#1\end{center}
    }
}

\setlength{\droptitle}{-2em}

  \title{Analysis of Mile per Gallon vs.~Tranmission via Regression Models}
    \pretitle{\vspace{\droptitle}\centering\huge}
  \posttitle{\par}
    \author{Aliakbar Safilian\footnote{Email:
  \href{mailto:a.a.safilian@gmail.com}{\nolinkurl{a.a.safilian@gmail.com}}}}
    \preauthor{\centering\large\emph}
  \postauthor{\par}
      \predate{\centering\large\emph}
  \postdate{\par}
    \date{January 21, 2019}

\usepackage{booktabs}
\usepackage{longtable}
\usepackage{array}
\usepackage{multirow}
\usepackage{wrapfig}
\usepackage{float}
\usepackage{colortbl}
\usepackage{pdflscape}
\usepackage{tabu}
\usepackage{threeparttable}
\usepackage{threeparttablex}
\usepackage[normalem]{ulem}
\usepackage{makecell}
\usepackage{xcolor}

\usepackage{color}
\usepackage{float}

\begin{document}
\maketitle

{
\hypersetup{linkcolor=black}
\setcounter{tocdepth}{2}
\tableofcontents
}
\newtheorem{theorem}{Theorem} \newtheorem{lemma}{Lemma}

\section{Overview}\label{overview}

In this report, we explore the relationship between a set of variables
and \emph{miles per gallon} (MPG). The dataset of interest in this
report is \textbf{mtcars} from the dataset package.

The data includes the following variables:

\begin{itemize}
\tightlist
\item
  \textbf{cyl}: Number of cylinders
\item
  \textbf{disp}: Displacement (cu.in.)
\item
  \textbf{hp}: Gross horsepower
\item
  \textbf{drat}: Rear axle ratio
\item
  \textbf{wt}: Weight (1000 lbs)
\item
  \textbf{qsec}: 1/4 mile time
\item
  \textbf{vs}: Engine shape, i.e., V-shaped or straight
\item
  \textbf{am}: Automatic (0) or manual (1) transimission
\item
  \textbf{gear}: Number of forward gears
\item
  \textbf{carb}: Number of carburetors
\end{itemize}

We are particularly interested in the following two questions:

\begin{itemize}
\tightlist
\item
  ``Is an automatic or manual transmission better for MPG''
\item
  ``Quantify the MPG difference between automatic and manual
  transmissions''
\end{itemize}

In \protect\hyperlink{sec-prelim}{Sect.2}, we do some prelimanry
analysis, including loading, transformation, and some summary and
exploratory analysis. We show that, in general, we expect that manual
transmission works better than automatic transmission with respect to
fuel economy. We deeply investigate this in the subsequent sections,
considering many other factors.

In \protect\hyperlink{sec-single}{Sect.3}, for each numeric variable
\(var\), we build a linear model with \textbf{mpg} as the output and
\(var\) as the regressor considering its interaction with the
transmission type. We first select linear models which are worth
considering. Then, we address our main questions using by these linear
models.

In \protect\hyperlink{sec-mult}{Sect.4}, we consider multivariate
regression models. Again, we select the best fitting models, and then
study them to address our analysis questions.

The R scripts of \protect\hyperlink{sec-prelim}{Sect.2},
\protect\hyperlink{sec-single}{Sect.3}, and
\protect\hyperlink{sec-mult}{Sect.4} can be found in
\protect\hyperlink{app-code-2}{Appendix.A},
\protect\hyperlink{app-code-3}{Appendix.B}, and
\protect\hyperlink{app-code-4}{Appendix.C}.
\protect\hyperlink{app-diag}{Appendix.D} include the diagnosis plots for
the fitting models.

We consider 0.05 as the significance rate in all statistical analyses in
this report.

\hypertarget{sec-prelim}{\section{Preliminary
Analysis}\label{sec-prelim}}

Let us first take a look at the structure of the data: (We transform
\textbf{cyl} (number of cylinders), \textbf{vs} (engine shape),
\textbf{am} (transmission), \textbf{gear}, and \textbf{carb} (number of
carburetors) to their equivalent factor variables. Also, we have renamed
the levels of the \textbf{am} and \textbf{vs} variables. Moreover, we
have changed the unit of \textbf{wt} from 1000lb to tonne.)

\begin{verbatim}
'data.frame':   32 obs. of  11 variables:
 $ mpg : num  21 21 22.8 21.4 18.7 18.1 14.3 24.4 22.8 19.2 ...
 $ cyl : Factor w/ 3 levels "4","6","8": 2 2 1 2 3 2 3 1 1 2 ...
 $ disp: num  160 160 108 258 360 ...
 $ hp  : num  110 110 93 110 175 105 245 62 95 123 ...
 $ drat: num  3.9 3.9 3.85 3.08 3.15 2.76 3.21 3.69 3.92 3.92 ...
 $ wt  : num  1.19 1.3 1.05 1.46 1.56 ...
 $ qsec: num  16.5 17 18.6 19.4 17 ...
 $ vs  : Factor w/ 2 levels "v","s": 1 1 2 2 1 2 1 2 2 2 ...
 $ am  : Factor w/ 2 levels "automatic","manual": 2 2 2 1 1 1 1 1 1 1 ...
 $ gear: Factor w/ 3 levels "3","4","5": 2 2 2 1 1 1 1 2 2 2 ...
 $ carb: Factor w/ 6 levels "1","2","3","4",..: 4 4 1 1 2 1 4 2 2 4 ...
\end{verbatim}

As we see in the boxplot in Fig. \ref{fig:boxplot}, in general, the
manual transformation is better than the automatic transmission in the
sense of fuel economy. The mean of MPG for automatic transmission and
manual transmission are 17.15 and 24.39, respectively.\\

\begin{figure}[H]

{\centering \includegraphics[width=0.7\linewidth]{D7_MPG_files/figure-latex/boxplot-1} 

}

\caption{\label{fig:boxplot}The Box Plot for MPG per Transmission Type}\label{fig:boxplot}
\end{figure}

\hypertarget{sec-single}{\section{Single Variate Regression
Models}\label{sec-single}}

In this section, for each numeric variable \emph{var}, we build a linear
model with MPG as the output and \emph{var} as the regressor considering
its interaction with the transmission type, i.e., the model \(mpg\)
\textasciitilde{} \(var * factor(am)\).

In \protect\hyperlink{sec-sing-select}{Sect.3.1}, we select the linear
models which are worth considering. In
\protect\hyperlink{sec-fit-wt}{Sect.3.2},
\protect\hyperlink{sec-fit-disp}{Sect.3.3}, and
\protect\hyperlink{sec-fit-hp}{Sect.3.4}, we address our main analysis
question on the selected models.

\hypertarget{sec-sing-select}{\subsection{Model
Selection}\label{sec-sing-select}}

We show that

\begin{theorem}
The best three single variate models with mpg as the output are mpg $\sim$ wt*am, mpg $\sim$ disp*am, and mpg $\sim$ hp*am. $\square$
\end{theorem}

The rest of this section is devoted to the proof of the theorem.

The correlation between the numeric variables in the dataset is shown in
the chart in Fig. \ref{fig:cor-chart}. As we see in the chart, the
correlation between \textbf{mpg} and \textbf{qsec}, i.e., 0.42, is not
that high. Therefore, we ignore the model fitting of \textbf{mpg} vs.
\textbf{qsec}.\\

\begin{figure}[H]

{\centering \includegraphics[width=0.7\linewidth]{D7_MPG_files/figure-latex/corchart-1} 

}

\caption{\label{fig:cor-chart}The Correlation Chart}\label{fig:corchart}
\end{figure}

Our models are as follow. Note that the weigth unit in the dataset is
1000lb; however, we change it to tonne.

\begin{Shaded}
\begin{Highlighting}[]
\NormalTok{fit_wt <-}\StringTok{ }\KeywordTok{lm}\NormalTok{(mpg }\OperatorTok{~}\StringTok{ }\NormalTok{wt }\OperatorTok{*}\StringTok{ }\KeywordTok{factor}\NormalTok{(am), }\DataTypeTok{data =}\NormalTok{ mtcars)}
\NormalTok{fit_disp <-}\StringTok{ }\KeywordTok{lm}\NormalTok{(mpg }\OperatorTok{~}\StringTok{ }\NormalTok{disp  }\OperatorTok{*}\StringTok{ }\KeywordTok{factor}\NormalTok{(am), }\DataTypeTok{data =}\NormalTok{ mtcars)}
\NormalTok{fit_hp <-}\StringTok{ }\KeywordTok{lm}\NormalTok{(mpg }\OperatorTok{~}\StringTok{ }\NormalTok{hp }\OperatorTok{*}\StringTok{ }\KeywordTok{factor}\NormalTok{(am), }\DataTypeTok{data =}\NormalTok{ mtcars)}
\NormalTok{fit_drat <-}\StringTok{ }\KeywordTok{lm}\NormalTok{(mpg }\OperatorTok{~}\StringTok{ }\NormalTok{drat }\OperatorTok{*}\StringTok{ }\KeywordTok{factor}\NormalTok{(am), }\DataTypeTok{data =}\NormalTok{ mtcars)}
\end{Highlighting}
\end{Shaded}

Since the above models are non-nested models, we take advantage of the
\emph{coxtest} function from the \emph{lmtest} package to test them. In
Table. \ref{tab:tab-pvalue}, each row and column belongs to a fitting
model. For any \(1 \leq i, j \leq 4\), the cell in the position
\([i,j]\) represents the P-value of comparing \(m_i\) against \(n_j\),
where (\(\forall t\)) \(m_t\) denotes the fitted model in row \(t\), and
\(n_t\) denotes the model represented in column \(j\).

\begin{table}[!h]

\caption{\label{tab:tab-pvalue}P-values of Comparing Single Variate Models by coxtest}
\centering
\begin{tabular}{lrrrr}
\toprule
  &    fit\_wt &    fit\_disp &    fit\_hp &    fit\_drat\\
\midrule
fit\_wt & 0 & 0.01 & 0.00 & 0.38\\
fit\_disp & 0 & 0.00 & 0.03 & 0.11\\
fit\_hp & 0 & 0.01 & 0.00 & 0.15\\
fit\_drat & 0 & 0.00 & 0.00 & 0.00\\
\bottomrule
\end{tabular}
\end{table}

Considering \(0.05\) as our significance rate, the results are as
follow:

\begin{itemize}
\item
  All other fitting models are preferred over the model fit\_drat.
\item
  The models other than fit\_drat have no preference over each other,
  though the best among all the models is fit\_wt.
\end{itemize}

Therefore, we discard the model fit\_drat.

One could find the corresponding diagnosis plots in
\protect\hyperlink{app-diag-1}{Appendix. 1}. As we see in the diagnostic
plots, everything (including normality of errors and residuals) looks
more less ok for the fitting models.

In the rest of this section, we investigate the fitting models excluding
\textbf{fit\_drat}.

\hypertarget{sec-fit-wt}{\subsection{MPG vs.~Weight}\label{sec-fit-wt}}

The first model that we study is fit\_wt, i.e., mpg \(\sim\) wt*am. Let
us take a look at its coefficients:

\begin{verbatim}
                    Estimate Std. Error t value Pr(>|t|)
(Intercept)            31.42       3.02   10.40        0
wt                     -8.35       1.73   -4.82        0
factor(am)manual       14.88       4.26    3.49        0
wt:factor(am)manual   -11.68       3.19   -3.67        0
\end{verbatim}

The 95\% confidence intervals of the coefficients are represented in the
following table.

\begin{table}[H]
\centering
\begin{tabular}{lrrrr}
\toprule
  & Intercept & wt & factor.am.manual & wt.factor.am.manual\\
\midrule
Lower & 25.23 & -11.89 & 6.15 & -18.21\\
Upper & 37.61 & -4.81 & 23.61 & -5.15\\
\bottomrule
\end{tabular}
\end{table}

The estimated intercept in automatic transmission is about 31.42, and
14.88 is the estimated change in the intercept of the linear
relationship between weigth and MPG going from automatic transmission to
manual transmission. The estimated slope in automatic transmission is
-8.35 while the estimated change in the slope switching from automatic
to manual is -11.68. In other words:

\begin{itemize}
\item
  The estimated MPG for automatic and manual vehicles with 0 weight are
  31.42 and 46.3, respectively.
\item
  The expected change in MPG per 1 tonne change in weight for automatic
  and manual vehicles are -8.35 and -20.03, respectively.
\end{itemize}

Fig. \ref{fig:fit_wt} represents the corresponding plot, where the
regression lines for automatic transmission and manaual transmission are
shown in red and blue, respectively.

\begin{figure}[H]

{\centering \includegraphics[width=0.7\linewidth]{D7_MPG_files/figure-latex/plot-fitwt-1} 

}

\caption{\label{fig:fit_wt}MPG vs. Weight}\label{fig:plot-fitwt}
\end{figure}

As it is clear heavier vehicle results in more fuel consumption. The
regression lines meets at point \textbf{1.27} tonne. As seen, we predict
that for vehicles with weight less (more, respectively) than 1.27 tonne,
the manual (automatic, respectively) transmission is a better for fuel
economy.

\hypertarget{sec-fit-disp}{\subsection{MPG
vs.~Displacement}\label{sec-fit-disp}}

The next model is fit\_disp, i.e., mpg \(\sim\) disp*am whose
coefficients are as follow:

\begin{verbatim}
                      Estimate Std. Error t value Pr(>|t|)
(Intercept)              25.16       1.93   13.07     0.00
disp                     -0.03       0.01   -4.44     0.00
factor(am)manual          7.71       2.50    3.08     0.00
disp:factor(am)manual    -0.03       0.01   -2.75     0.01
\end{verbatim}

The 95\% confidence intervals of the coefficients are represented in the
following table.

\begin{table}[H]
\centering
\begin{tabular}{lrrrr}
\toprule
  & Intercept & disp & factor.am.manual & disp.factor.am.manual\\
\midrule
Lower & 21.21 & -0.05 & 2.59 & -0.05\\
Upper & 29.11 & -0.01 & 12.83 & -0.01\\
\bottomrule
\end{tabular}
\end{table}

The estimated intercept in automatic transmission is about 25.16, while
7.71 is the estimated change in the intercept of the linear relationship
between displacement and MPG going from automatic transmission to manual
transmission. The estimated slope in automatic transmission is -0.03
while the estimated change in the slope switching from automatic to
manual is -0.03. In other words:

\begin{itemize}
\item
  The estimated MPG for automatic transmission and manual transmission
  vehicles with 0 cu.in. displacement are 25.16 and 32.87, respectively.
\item
  The expected change in MPG per 1 cu.in. change in displacement for
  automatic and manual vehicles are -0.03 and -0.06, respectively.
\end{itemize}

Fig. \ref{fig:fit_disp} the corresponding plot, where the regression
lines for automatic transmission and manaual transmission are shown in
blue and red, respectively.

\begin{figure}[H]

{\centering \includegraphics[width=0.7\linewidth]{D7_MPG_files/figure-latex/plot-fitdisp-1} 

}

\caption{\label{fig:fit_disp}MPG vs. Displacement}\label{fig:plot-fitdisp}
\end{figure}

As it is clear, higher displacement results in more fuel consumption.
The regression lines meets at point \textbf{257} cu.in. As seen in the
plot, we predict that for vehicles with displacement less (more,
respectively) than 257 (cu.in.), the manual (automatic, respectively)
transmission workes better w.r.t fuel economy.

\hypertarget{sec-fit-hp}{\subsection{MPG
vs.~Horsepower}\label{sec-fit-hp}}

The last model to study is fit\_hp, i.e., mpg \(\sim\) hp*am with the
following coefficients:

\begin{verbatim}
                    Estimate Std. Error t value Pr(>|t|)
(Intercept)            26.62       2.18   12.20     0.00
hp                     -0.06       0.01   -4.57     0.00
factor(am)manual        5.22       2.67    1.96     0.06
hp:factor(am)manual     0.00       0.02    0.02     0.98
\end{verbatim}

The 95\% confidence intervals of the coefficients are represented in the
following table.

\begin{table}[H]
\centering
\begin{tabular}{lrrrr}
\toprule
  & Intercept & hp & factor.am.manual & hp.factor.am.manual\\
\midrule
Lower & 22.15 & -0.08 & -0.25 & -0.04\\
Upper & 31.09 & -0.04 & 10.69 & 0.04\\
\bottomrule
\end{tabular}
\end{table}

The estimated intercept in automatic transmission is about 26.62. 5.22
is the estimated change in the intercept of the linear relationship
between horsepower and MPG going from automatic transmission to manual
transmission. The estimated slope in automatic transmission is -0.06
while the estimated change in the slope switching from automatic to
manual is about 0. This shows that there is no signigicant interaction
between \textbf{am} and \textbf{hp}. In other words:

\begin{itemize}
\tightlist
\item
  The expected MPG for automatic transmission and manual transmission
  vehicles with horsepower 0 are 26.62 and 31.84, respectively.\\
\item
  The expected change in MPG per unit change in horsepower for both
  automatic and manual vehicles is about -0.06.
\end{itemize}

Note that the second bullet implies that the corresponding regression
lines for automatic and manual would be parallel. Fig. \ref{fig:fit_hp}
represents the corresponding plot, where the regression lines for
automatic transmission and manaual transmission are shown in red and
blue, respectively.

\begin{figure}[H]

{\centering \includegraphics[width=0.7\linewidth]{D7_MPG_files/figure-latex/plot-fithp-1} 

}

\caption{\label{fig:fit_hp}MPG vs. Horsepower}\label{fig:plot-fithp}
\end{figure}

Clearly, higher horsepower results in worse fuel economy. As seen in the
above plot, we predict that manual transmission is always better than
automatic transmission with respect to fuel economy for a given
horsepower.

\hypertarget{sec-mult}{\section{Multivariate Regression
Models}\label{sec-mult}}

In this section, we consider more complicated models, i.e., multivariate
regression models.

The structure of this section is as follows: In
\protect\hyperlink{sec-mult-select}{Sect.4.1}, we select the
multivariate linear regression models which are worth considering. In
\protect\hyperlink{sec-fit-wthp}{Sect.4.2} and
\protect\hyperlink{sec-fit-wtqsec}{Sect.4.3}, we address our main
analysis question on the selected models.

\hypertarget{sec-mult-select}{\subsection{Model
Selection}\label{sec-mult-select}}

We show that:

\begin{theorem}
The best two fitting linear models with mpg as the output are mpg $\sim$ wt+hp and mpg $\sim$ wt+qsec. $\square$
\end{theorem}

As we already saw, among the single variate models, the models worth to
consider are \textbf{Model1}: mpg\textasciitilde{}wt, \textbf{Model2}:
mpg\textasciitilde{}hp, and \textbf{Model3}: mpg\textasciitilde{}disp.
Now, we want to see if adding some new regressors to these models make
sense. We show that

\begin{lemma}\label{lem:lem1}
The best models including wt as a regressor are mpg $\sim$ wt+hp and mpg $\sim$ wt+qsec. $\square$ 
\end{lemma}

\begin{lemma}\label{lem:lem2}
The best model including hp as a must regressor is mpg $\sim$ hp+wt.  $\square$
\end{lemma}

\begin{lemma}\label{lem:lem3}
Considering disp as a must regressor in our models, the best linear model is mpg $\sim$ disp+wt. $\square$ 
\end{lemma}

Note that these three lemmas together prove our main theorem.

\subsubsection{\texorpdfstring{Proof of Lemma.
\ref{lem:lem1}}{Proof of Lemma. }}\label{proof-of-lemma.}

Let us first see if adding some regressors to mpg\textasciitilde{}wt
makes senses. We take advantage of the \emph{anova} function to address
this question. The P-values for comparing the model \(mpg\)
\textasciitilde{} \(wt\) vs. \(mpg\) \textasciitilde{} \(wt + var\),
where \(var \in \{disp, hp, drat, qsec\}\) are represented in Table.
\ref{tab:test-mult-wt}.

\begin{table}[!h]

\caption{\label{tab:test-mult-wt}P-values of Comparing 2-variate Models with wt as the Regressor}
\centering
\begin{tabular}{lrrrr}
\toprule
  & mpg\textasciitilde{}wt+disp &   mpg\textasciitilde{}wt+hp &   mpg\textasciitilde{}wt+drat &   mpg\textasciitilde{}wt+qsec\\
\midrule
mpg\textasciitilde{}wt & 0.06 & 0 & 0.33 & 0\\
\bottomrule
\end{tabular}
\end{table}

Considering the significance rate 0.05, we see that only the two models
``mpg \textasciitilde{} wt + hp'' and ``mpg \textasciitilde{} wt +
qsec'' are preferred over mpg \textasciitilde{} wt. Now, let us see
which of these two models works better:

\begin{verbatim}
Cox test

Model 1: mpg ~ wt + hp
Model 2: mpg ~ wt + qsec
                Estimate Std. Error z value Pr(>|z|)
fitted(M1) ~ M2    -2.25       1.54   -1.46     0.14
fitted(M2) ~ M1    -2.30       1.53   -1.50     0.13
\end{verbatim}

Therefore, considering 0.05 as the significance rate, none of them are
preferred over the other.

Now, let us see if their combination, i.e., mpg \textasciitilde{} wt +
hp + qsec, works better. In the following, we show that mpg
\textasciitilde{} wt + hp + qsec is NOT preferred over mpg
\textasciitilde{} wt + hp.

\begin{verbatim}
Analysis of Variance Table

Model 1: mpg ~ wt + hp
Model 2: mpg ~ wt + hp + qsec
  Res.Df    RSS Df Sum of Sq      F Pr(>F)
1     29 195.05                           
2     28 186.06  1    8.9885 1.3527 0.2546
\end{verbatim}

Now, we show that mpg \textasciitilde{} wt + hp + qsec is NOT preferred
over mpg \textasciitilde{} wt + qsec:

\begin{verbatim}
Analysis of Variance Table

Model 1: mpg ~ wt + qsec
Model 2: mpg ~ wt + hp + qsec
  Res.Df    RSS Df Sum of Sq      F Pr(>F)
1     29 195.46                           
2     28 186.06  1    9.4043 1.4153 0.2442
\end{verbatim}

Therefore, we showed that the best models including \textbf{wt} as a
must regressor is \emph{mpg \textasciitilde{} wt + hp} or \emph{mpg
\textasciitilde{} wt + qsec}. Lemma. \ref{lem:lem1} was proven.

\subsubsection{\texorpdfstring{Proof of Lemma.
\ref{lem:lem2}}{Proof of Lemma. }}\label{proof-of-lemma.-1}

Let us now see if adding some regressors to mpg \textasciitilde{} hp
makes our fitting model any better. Again, we apply the \emph{anova}
function to address this question. The P-value for comparing the model
mpg\textasciitilde{}hp vs.~mpg\textasciitilde{}hp+\(var\), where
\(var \in \{disp, wt, drat, qsec\}\) are represented in Table.
\ref{tab:test-mult-hp}.

\begin{table}[!h]

\caption{\label{tab:test-mult-hp}P-values of Comparing 2-variate Models with hp as the Regressor}
\centering
\begin{tabular}{lrrrr}
\toprule
  & mpg\textasciitilde{}hp+disp &   mpg\textasciitilde{}hp+wt &   mpg\textasciitilde{}hp+drat &   mpg\textasciitilde{}hp+qsec\\
\midrule
mpg\textasciitilde{}hp & 0 & 0 & 0 & 0.11\\
\bottomrule
\end{tabular}
\end{table}

Considering the significance rate 0.05, all the models excluding
mpg\textasciitilde{}hp+qsec are preferred over mpg\textasciitilde{}hp.
Now, using the \emph{coxtest} command, we want to see which of them
works the best:

The following tables show that \emph{mpg\textasciitilde{}hp+wt} has
preference over mpg\textasciitilde{}hp+disp and
mpg\textasciitilde{}hp+drat:

\begin{verbatim}
Cox test

Model 1: mpg ~ hp + disp
Model 2: mpg ~ hp + wt
                Estimate Std. Error z value  Pr(>|z|)    
fitted(M1) ~ M2  -9.0487     1.7108 -5.2892 1.229e-07 ***
fitted(M2) ~ M1  -0.2062     2.1047 -0.0980     0.922    
---
Signif. codes:  0 '***' 0.001 '**' 0.01 '*' 0.05 '.' 0.1 ' ' 1
\end{verbatim}

\begin{verbatim}
Cox test

Model 1: mpg ~ hp + wt
Model 2: mpg ~ hp + drat
                Estimate Std. Error z value  Pr(>|z|)    
fitted(M1) ~ M2  -2.9554     1.7423 -1.6963   0.08983 .  
fitted(M2) ~ M1 -10.9891     1.5147 -7.2548 4.023e-13 ***
---
Signif. codes:  0 '***' 0.001 '**' 0.01 '*' 0.05 '.' 0.1 ' ' 1
\end{verbatim}

Therefore, we showed that the best model including \textbf{hp} as a must
regressor is \emph{mpg\textasciitilde{}hp+wt}. Lemma. \ref{lem:lem2} is
proven.

\subsubsection{\texorpdfstring{Proof of Lemma.
\ref{lem:lem3}}{Proof of Lemma. }}\label{proof-of-lemma.-2}

The last model to be considered is mpg\textasciitilde{}disp. Applying
the \emph{anova} function, the P-values for comparing the model
mpg\textasciitilde{}disp vs.~mpg\textasciitilde{}disp+\(var\), where
\(var \in \{hp, wt, drat, qsec\}\) are represented in Table.
\ref{tab:test-mult-disp}.

As we see in the table, \emph{mpg\textasciitilde{}disp+wt} is the only
model which is preferred over mpg\textasciitilde{}disp. In other words,
considering disp as a must regressor in our models, the best linear
model is \emph{mpg\textasciitilde{}disp+wt}. Lemma. \ref{lem:lem3} is
proven.

\begin{table}[!h]

\caption{\label{tab:test-mult-disp}P-values of Comparing 2-variate Models with disp as the Regressor}
\centering
\begin{tabular}{lrrrr}
\toprule
  & mpg\textasciitilde{}disp+hp &   mpg\textasciitilde{}disp+wt &   mpg\textasciitilde{}disp+drat &   mpg\textasciitilde{}dispp+qsec\\
\midrule
mpg\textasciitilde{}disp & 0.07 & 0.01 & 0.25 & 0.57\\
\bottomrule
\end{tabular}
\end{table}

\hypertarget{sec-fit-wthp}{\subsection{MPG vs.~Weight plus
Horsepower}\label{sec-fit-wthp}}

In this section, we study the model with \textbf{wt} and \textbf{hp} as
regressors. Let's first consider the full interaction between
transmission type and the regressors, i.e., the following model:

\begin{Shaded}
\begin{Highlighting}[]
\NormalTok{fit_wt_hp <-}\StringTok{ }\KeywordTok{lm}\NormalTok{(mpg }\OperatorTok{~}\StringTok{ }\NormalTok{(wt }\OperatorTok{+}\StringTok{ }\NormalTok{hp) }\OperatorTok{*}\StringTok{ }\KeywordTok{factor}\NormalTok{(am), }\DataTypeTok{data =}\NormalTok{ mtcars)}
\end{Highlighting}
\end{Shaded}

The coefficients of the model are as follow:

\begin{verbatim}
                    Estimate Std. Error t value Pr(>|t|)
(Intercept)            30.70       2.68   11.48     0.00
wt                     -4.09       2.08   -1.96     0.06
hp                     -0.04       0.01   -3.00     0.01
factor(am)manual       13.74       4.22    3.25     0.00
wt:factor(am)manual   -12.72       4.57   -2.78     0.01
hp:factor(am)manual     0.03       0.02    1.45     0.16
\end{verbatim}

As we see, the P-value for the interaction of \textbf{hp} and
\textbf{am} is not significant. Moreover, the P-value for the
coefficient \textbf{wt} is a little higher than the sigficance rate
(0.05). Therefore, we modify the model as follows:

\begin{Shaded}
\begin{Highlighting}[]
\NormalTok{fit_wt_hp <-}\StringTok{ }\KeywordTok{lm}\NormalTok{(mpg }\OperatorTok{~}\StringTok{ }\NormalTok{(wt }\OperatorTok{*}\StringTok{ }\KeywordTok{factor}\NormalTok{(am)) }\OperatorTok{+}\StringTok{ }\NormalTok{hp, }\DataTypeTok{data =}\NormalTok{ mtcars)}
\end{Highlighting}
\end{Shaded}

The coefficients of the model are as follow:

\begin{verbatim}
                    Estimate Std. Error t value Pr(>|t|)
(Intercept)            30.95       2.72   11.36     0.00
wt                     -5.55       1.86   -2.98     0.01
factor(am)manual       11.55       4.02    2.87     0.01
hp                     -0.03       0.01   -2.75     0.01
wt:factor(am)manual    -7.89       3.18   -2.48     0.02
\end{verbatim}

The 95\% confidence intervals of the coefficients are represented in the
following table.

\begin{table}[H]
\centering
\begin{tabular}{lrrrrr}
\toprule
  & Intercept & wt & factor.am.manual & hp & wt.factor.am.manual\\
\midrule
Lower & 25.37 & -9.37 & 3.3 & -0.05 & -14.41\\
Upper & 36.53 & -1.73 & 19.8 & -0.01 & -1.37\\
\bottomrule
\end{tabular}
\end{table}

The estimated intercept in automatic transmission is about 30.95, and
11.55 is the estimated change in the intercept of the linear
relationship going from automatic transmission to manual transmission.
The estimated coefficient of weigth in automatic transmission is -5.55
while the estimated change in the weigth coefficient switching from
automatic to manual is -7.89. The estimated coefficient of horsepower is
-0.03.

In other words:

\begin{itemize}
\item
  The estimated MPG is 30.95 and 42.5 for the automatic transmission and
  the manual transmission vehicle with weight 0 and horsepower 0,
  respectively.
\item
  The expected change in MPG for an automatic transmission and manual
  transmission per tonne change in weight are -5.55 and -13.44,
  respectively, by holding the horsepwer constant.
\item
  The expected change in MPG for both automatic and manual vehicle per
  unit change in horsepower is -0.03, by holding weight constant.
\end{itemize}

We have represented the diagnosis plots of this model in Fig.
\ref{fig:diag-2-hp}.

\hypertarget{sec-fit-wtqsec}{\subsection{MPG vs.~Weight plus
1/4-Mile-Time}\label{sec-fit-wtqsec}}

In this section, we study the model with \textbf{wt} and \textbf{qsec}
as regressors. We first consider the full interaction between
transmission type and the regressors, i.e., the following model:

\begin{Shaded}
\begin{Highlighting}[]
\NormalTok{fit_wt_qsec <-}\StringTok{ }\KeywordTok{lm}\NormalTok{(mpg }\OperatorTok{~}\StringTok{ }\NormalTok{(wt }\OperatorTok{+}\StringTok{ }\NormalTok{qsec) }\OperatorTok{*}\StringTok{ }\KeywordTok{factor}\NormalTok{(am), }\DataTypeTok{data =}\NormalTok{ mtcars)}
\end{Highlighting}
\end{Shaded}

The coefficients of the model are as follow:

\begin{verbatim}
                      Estimate Std. Error t value Pr(>|t|)
(Intercept)              11.25       6.99    1.61     0.12
wt                       -6.61       1.52   -4.34     0.00
qsec                      0.95       0.31    3.08     0.00
factor(am)manual          8.93      12.67    0.70     0.49
wt:factor(am)manual      -8.29       3.34   -2.48     0.02
qsec:factor(am)manual     0.24       0.56    0.42     0.68
\end{verbatim}

As we see, the P-value for the interaction of \textbf{qsec} and
\textbf{am} is not significant. Therefore, we modify the model as
follows:

\begin{Shaded}
\begin{Highlighting}[]
\NormalTok{fit_wt_qsec <-}\StringTok{ }\KeywordTok{lm}\NormalTok{(mpg }\OperatorTok{~}\StringTok{ }\NormalTok{(wt }\OperatorTok{*}\StringTok{ }\KeywordTok{factor}\NormalTok{(am)) }\OperatorTok{+}\StringTok{ }\NormalTok{qsec, }\DataTypeTok{data =}\NormalTok{ mtcars)}
\end{Highlighting}
\end{Shaded}

The coefficients of the new model are as follow:

\begin{verbatim}
                    Estimate Std. Error t value Pr(>|t|)
(Intercept)             9.72       5.90    1.65     0.11
wt                     -6.47       1.47   -4.41     0.00
factor(am)manual       14.08       3.44    4.10     0.00
qsec                    1.02       0.25    4.04     0.00
wt:factor(am)manual    -9.13       2.64   -3.46     0.00
\end{verbatim}

The 95\% confidence intervals of the coefficients are represented in the
following table.

\begin{table}[H]
\centering
\begin{tabular}{lrrrrr}
\toprule
  & Intercept & wt & factor.am.manual & qsec & wt.factor.am.manual\\
\midrule
Lower & -2.39 & -9.49 & 7.02 & 0.51 & -14.55\\
Upper & 21.83 & -3.45 & 21.14 & 1.53 & -3.71\\
\bottomrule
\end{tabular}
\end{table}

Since the P-value associated with intercept is high, our estimated
intercept in automatic transmission is 0. 14.08 is the estimated change
in the intercept of the linear relationship going from automatic
transmission to manual transmission. The estimated coefficient of weigth
in automatic transmission is -6.47 while the estimated change in the
weigth coefficient switching from automatic to manual is -9.13. The
estimated coefficient of qsec is 1.02.

In other words:

\begin{itemize}
\item
  The estimated MPG is 0 for an automatic transmission vehicle with
  weight 0 and qsect 0.
\item
  The estimated MPG is 14.08 for a manual transmission vehicle with
  weight 0 and qsect 0.
\item
  The expected change in MPG for an automatic transmission vehicle per
  tonne change in weight is -6.47, by holding the qsec constant.
\item
  The expected change in MPG for an automatic manual vehicle per tonne
  change in weight is -15.6, by holding the qsec constant.
\item
  The expected change in MPG for both automatic and manual vehicle per
  unit change in qsec is 1.02, by holding weight constant
\end{itemize}

The diagnosis plot of this models can be found in Fig.
\ref{fig:diag-2-qsec}.

\hypertarget{app-code-2}{\section*{\texorpdfstring{Appendix A: R Scripts
of \protect\hyperlink{sec-prelim}{Sect.
2}}{Appendix A: R Scripts of Sect. 2}}\label{app-code-2}}
\addcontentsline{toc}{section}{Appendix A: R Scripts of
\protect\hyperlink{sec-prelim}{Sect. 2}}

Loading and transforming the data:

\begin{Shaded}
\begin{Highlighting}[]
\KeywordTok{library}\NormalTok{(datasets)}
\KeywordTok{data}\NormalTok{(}\StringTok{"mtcars"}\NormalTok{)}
\NormalTok{mtcars}\OperatorTok{$}\NormalTok{am <-}\StringTok{ }\KeywordTok{as.factor}\NormalTok{(mtcars}\OperatorTok{$}\NormalTok{am)}
\KeywordTok{levels}\NormalTok{(mtcars}\OperatorTok{$}\NormalTok{am) <-}\StringTok{ }\KeywordTok{c}\NormalTok{(}\StringTok{"automatic"}\NormalTok{, }\StringTok{"manual"}\NormalTok{)}
\NormalTok{mtcars}\OperatorTok{$}\NormalTok{cyl <-}\StringTok{ }\KeywordTok{as.factor}\NormalTok{(mtcars}\OperatorTok{$}\NormalTok{cyl)}
\NormalTok{mtcars}\OperatorTok{$}\NormalTok{vs <-}\StringTok{ }\KeywordTok{as.factor}\NormalTok{(mtcars}\OperatorTok{$}\NormalTok{vs)}
\KeywordTok{levels}\NormalTok{(mtcars}\OperatorTok{$}\NormalTok{vs) <-}\StringTok{ }\KeywordTok{c}\NormalTok{(}\StringTok{"v"}\NormalTok{, }\StringTok{"s"}\NormalTok{)}
\NormalTok{mtcars}\OperatorTok{$}\NormalTok{gear <-}\StringTok{ }\KeywordTok{as.factor}\NormalTok{(mtcars}\OperatorTok{$}\NormalTok{gear)}
\NormalTok{mtcars}\OperatorTok{$}\NormalTok{carb <-}\StringTok{ }\KeywordTok{as.factor}\NormalTok{(mtcars}\OperatorTok{$}\NormalTok{carb)}
\NormalTok{mtcars}\OperatorTok{$}\NormalTok{wt <-}\StringTok{ }\NormalTok{mtcars}\OperatorTok{$}\NormalTok{wt}\OperatorTok{/}\FloatTok{2.20462}
\KeywordTok{str}\NormalTok{(mtcars)}
\end{Highlighting}
\end{Shaded}

The boxplot for MPG per each transmission type:

\begin{Shaded}
\begin{Highlighting}[]
\KeywordTok{library}\NormalTok{(ggplot2)}
\KeywordTok{qplot}\NormalTok{(am, mpg, }\DataTypeTok{data =}\NormalTok{ mtcars, }\DataTypeTok{colour =}\NormalTok{ am) }\OperatorTok{+}\StringTok{ }\KeywordTok{geom_boxplot}\NormalTok{() }\OperatorTok{+}
\StringTok{        }\KeywordTok{xlab}\NormalTok{(}\StringTok{"Transmission"}\NormalTok{) }\OperatorTok{+}\StringTok{ }\KeywordTok{ylab}\NormalTok{(}\StringTok{"MPG"}\NormalTok{)}
\end{Highlighting}
\end{Shaded}

\hypertarget{app-code-3}{\section*{\texorpdfstring{Appendix B: R Scripts
of \protect\hyperlink{sec-single}{Sect.
3}}{Appendix B: R Scripts of Sect. 3}}\label{app-code-3}}
\addcontentsline{toc}{section}{Appendix B: R Scripts of
\protect\hyperlink{sec-single}{Sect. 3}}

The correlation chart:

\begin{Shaded}
\begin{Highlighting}[]
\KeywordTok{library}\NormalTok{(corrplot)}
\KeywordTok{library}\NormalTok{(PerformanceAnalytics)}
\KeywordTok{chart.Correlation}\NormalTok{(mtcars[, }\KeywordTok{c}\NormalTok{(}\DecValTok{1}\NormalTok{, }\DecValTok{3}\OperatorTok{:}\DecValTok{7}\NormalTok{)], }\DataTypeTok{histogram=}\OtherTok{TRUE}\NormalTok{, }\DataTypeTok{pch=}\DecValTok{19}\NormalTok{)}
\end{Highlighting}
\end{Shaded}

Code of Table. \ref{tab:tab-pvalue} (P-values of Comparing Single
Variate Models)

\begin{Shaded}
\begin{Highlighting}[]
\KeywordTok{library}\NormalTok{(lmtest)}
\NormalTok{wt_disp <-}\StringTok{ }\KeywordTok{round}\NormalTok{(}\KeywordTok{coxtest}\NormalTok{(fit_wt, fit_disp)}\OperatorTok{$}\StringTok{`}\DataTypeTok{Pr(>|z|)}\StringTok{`}\NormalTok{, }\DecValTok{2}\NormalTok{)}
\NormalTok{wt_hp <-}\StringTok{ }\KeywordTok{round}\NormalTok{(}\KeywordTok{coxtest}\NormalTok{(fit_wt, fit_hp)}\OperatorTok{$}\StringTok{`}\DataTypeTok{Pr(>|z|)}\StringTok{`}\NormalTok{, }\DecValTok{2}\NormalTok{)}
\NormalTok{wt_drat <-}\StringTok{ }\KeywordTok{round}\NormalTok{(}\KeywordTok{coxtest}\NormalTok{(fit_wt, fit_drat)}\OperatorTok{$}\StringTok{`}\DataTypeTok{Pr(>|z|)}\StringTok{`}\NormalTok{, }\DecValTok{2}\NormalTok{)}
\NormalTok{disp_hp <-}\StringTok{ }\KeywordTok{round}\NormalTok{(}\KeywordTok{coxtest}\NormalTok{(fit_disp, fit_hp)}\OperatorTok{$}\StringTok{`}\DataTypeTok{Pr(>|z|)}\StringTok{`}\NormalTok{, }\DecValTok{2}\NormalTok{)}
\NormalTok{disp_drat <-}\StringTok{ }\KeywordTok{round}\NormalTok{(}\KeywordTok{coxtest}\NormalTok{(fit_disp, fit_drat)}\OperatorTok{$}\StringTok{`}\DataTypeTok{Pr(>|z|)}\StringTok{`}\NormalTok{, }\DecValTok{2}\NormalTok{)}
\NormalTok{hp_drat <-}\StringTok{ }\KeywordTok{round}\NormalTok{(}\KeywordTok{coxtest}\NormalTok{(fit_hp, fit_drat)}\OperatorTok{$}\StringTok{`}\DataTypeTok{Pr(>|z|)}\StringTok{`}\NormalTok{, }\DecValTok{2}\NormalTok{)}
\NormalTok{wt_c <-}\StringTok{ }\KeywordTok{c}\NormalTok{(}\DecValTok{0}\NormalTok{, wt_disp[}\DecValTok{2}\NormalTok{], wt_hp[}\DecValTok{2}\NormalTok{], wt_drat[}\DecValTok{2}\NormalTok{])}
\NormalTok{disp_c <-}\StringTok{ }\KeywordTok{c}\NormalTok{(wt_disp[}\DecValTok{1}\NormalTok{], }\DecValTok{0}\NormalTok{, disp_hp[}\DecValTok{2}\NormalTok{], disp_drat[}\DecValTok{2}\NormalTok{])}
\NormalTok{hp_c <-}\StringTok{ }\KeywordTok{c}\NormalTok{(wt_hp[}\DecValTok{1}\NormalTok{], disp_hp[}\DecValTok{1}\NormalTok{], }\DecValTok{0}\NormalTok{, hp_drat[}\DecValTok{2}\NormalTok{])}
\NormalTok{drat_c <-}\StringTok{ }\KeywordTok{c}\NormalTok{(wt_drat[}\DecValTok{1}\NormalTok{], disp_drat[}\DecValTok{1}\NormalTok{], hp_drat[}\DecValTok{1}\NormalTok{], }\DecValTok{0}\NormalTok{)}
\NormalTok{tests_pval <-}\StringTok{ }\KeywordTok{as.data.frame}\NormalTok{(}\KeywordTok{cbind}\NormalTok{(wt_c, disp_c, hp_c, drat_c))}
\KeywordTok{row.names}\NormalTok{(tests_pval) <-}\StringTok{ }\KeywordTok{c}\NormalTok{(}\StringTok{"fit_wt"}\NormalTok{, }\StringTok{"fit_disp"}\NormalTok{, }\StringTok{"fit_hp"}\NormalTok{, }\StringTok{"fit_drat"}\NormalTok{)}
\KeywordTok{colnames}\NormalTok{(tests_pval) <-}\StringTok{ }\KeywordTok{paste}\NormalTok{(}\StringTok{"   "}\NormalTok{, }\KeywordTok{row.names}\NormalTok{(tests_pval), }\DataTypeTok{sep =} \StringTok{""}\NormalTok{)}

\KeywordTok{library}\NormalTok{(knitr)}
\KeywordTok{library}\NormalTok{(kableExtra)}
\KeywordTok{kable}\NormalTok{(tests_pval, }\StringTok{"latex"}\NormalTok{, }\DataTypeTok{booktabs =}\NormalTok{ T,  }
      \DataTypeTok{caption =} \StringTok{"P-values of Comparing Single Variate Models by coxtest"}\NormalTok{) }\OperatorTok
\StringTok{        }\KeywordTok{kable_styling}\NormalTok{(}\DataTypeTok{latex_options =} \StringTok{"hold_position"}\NormalTok{)}
\end{Highlighting}
\end{Shaded}

The coefficients of fit\_wt:

\begin{Shaded}
\begin{Highlighting}[]
\NormalTok{cff_wt <-}\StringTok{ }\KeywordTok{round}\NormalTok{(}\KeywordTok{summary}\NormalTok{(fit_wt)}\OperatorTok{$}\NormalTok{coefficient, }\DecValTok{2}\NormalTok{)}
\NormalTok{cff_wt}
\end{Highlighting}
\end{Shaded}

The 95\% confidence interval for the fit\_wt coefficients:

\begin{Shaded}
\begin{Highlighting}[]
\NormalTok{q <-}\StringTok{ }\KeywordTok{c}\NormalTok{(}\OperatorTok{-}\DecValTok{1}\NormalTok{,}\DecValTok{1}\NormalTok{) }\OperatorTok{*}\StringTok{ }\KeywordTok{qt}\NormalTok{(.}\DecValTok{975}\NormalTok{, }\DataTypeTok{df =}\NormalTok{ fit_wt}\OperatorTok{$}\NormalTok{df)}
\NormalTok{int_aut <-}\StringTok{ }\NormalTok{cff_wt[}\DecValTok{1}\NormalTok{,}\DecValTok{1}\NormalTok{] }\OperatorTok{+}\StringTok{ }\NormalTok{q }\OperatorTok{*}\StringTok{ }\NormalTok{cff_wt[}\DecValTok{1}\NormalTok{,}\DecValTok{2}\NormalTok{] }
\NormalTok{int_aut2man <-}\StringTok{ }\NormalTok{cff_wt[}\DecValTok{3}\NormalTok{,}\DecValTok{1}\NormalTok{] }\OperatorTok{+}\StringTok{ }\NormalTok{q }\OperatorTok{*}\StringTok{ }\NormalTok{cff_wt[}\DecValTok{3}\NormalTok{,}\DecValTok{2}\NormalTok{] }
\NormalTok{wt_aut <-}\StringTok{ }\NormalTok{cff_wt[}\DecValTok{2}\NormalTok{,}\DecValTok{1}\NormalTok{] }\OperatorTok{+}\StringTok{ }\NormalTok{q }\OperatorTok{*}\StringTok{ }\NormalTok{cff_wt[}\DecValTok{2}\NormalTok{,}\DecValTok{2}\NormalTok{] }
\NormalTok{wt_aut2man <-}\StringTok{ }\NormalTok{cff_wt[}\DecValTok{4}\NormalTok{,}\DecValTok{1}\NormalTok{] }\OperatorTok{+}\StringTok{ }\NormalTok{q }\OperatorTok{*}\StringTok{ }\NormalTok{cff_wt[}\DecValTok{4}\NormalTok{,}\DecValTok{2}\NormalTok{] }
 

\NormalTok{conf_wt <-}\StringTok{ }\KeywordTok{data.frame}\NormalTok{(}\DataTypeTok{Intercept =}\NormalTok{ int_aut, }
                          \DataTypeTok{wt =}\NormalTok{ wt_aut,  }
                          \StringTok{`}\DataTypeTok{factor(am)manual}\StringTok{`}\NormalTok{ =}\StringTok{ }\NormalTok{int_aut2man, }
                          \StringTok{`}\DataTypeTok{wt:factor(am)manual}\StringTok{`}\NormalTok{ =}\StringTok{ }\NormalTok{wt_aut2man)}

\KeywordTok{row.names}\NormalTok{(conf_wt) <-}\StringTok{ }\KeywordTok{c}\NormalTok{(}\StringTok{"Lower"}\NormalTok{, }\StringTok{"Upper"}\NormalTok{)}

\KeywordTok{kable}\NormalTok{(}\KeywordTok{round}\NormalTok{(conf_wt, }\DecValTok{2}\NormalTok{), }\StringTok{"latex"}\NormalTok{, }\DataTypeTok{booktabs =}\NormalTok{ T) }\OperatorTok
\StringTok{  }\KeywordTok{kable_styling}\NormalTok{(}\DataTypeTok{position =} \StringTok{"center"}\NormalTok{)}
\end{Highlighting}
\end{Shaded}

The corresponding plot for fit\_wt

\begin{Shaded}
\begin{Highlighting}[]
\NormalTok{line <-}\StringTok{ }\KeywordTok{data.frame}\NormalTok{(}\DataTypeTok{intercept =} \KeywordTok{c}\NormalTok{( cff_wt[}\DecValTok{1}\NormalTok{, }\DecValTok{1}\NormalTok{], cff_wt[}\DecValTok{1}\NormalTok{, }\DecValTok{1}\NormalTok{] }\OperatorTok{+}\StringTok{ }\NormalTok{cff_wt[}\DecValTok{3}\NormalTok{, }\DecValTok{1}\NormalTok{]),}
                   \DataTypeTok{slope =} \KeywordTok{c}\NormalTok{( cff_wt[}\DecValTok{2}\NormalTok{, }\DecValTok{1}\NormalTok{], cff_wt[}\DecValTok{2}\NormalTok{, }\DecValTok{1}\NormalTok{] }\OperatorTok{+}\StringTok{ }\NormalTok{cff_wt[}\DecValTok{4}\NormalTok{, }\DecValTok{1}\NormalTok{]), }
                   \DataTypeTok{row.names =} \KeywordTok{c}\NormalTok{(}\StringTok{"automatic"}\NormalTok{, }\StringTok{"manual"}\NormalTok{) )}
\NormalTok{x_com_wt <-}\StringTok{ }\NormalTok{(line[}\DecValTok{1}\NormalTok{,}\DecValTok{1}\NormalTok{] }\OperatorTok{-}\StringTok{ }\NormalTok{line[}\DecValTok{2}\NormalTok{,}\DecValTok{1}\NormalTok{]) }\OperatorTok{/}\StringTok{ }\NormalTok{(line[}\DecValTok{2}\NormalTok{,}\DecValTok{2}\NormalTok{] }\OperatorTok{-}\StringTok{ }\NormalTok{line[}\DecValTok{1}\NormalTok{, }\DecValTok{2}\NormalTok{]) }
\KeywordTok{qplot}\NormalTok{(wt, mpg, }\DataTypeTok{data =}\NormalTok{ mtcars) }\OperatorTok{+}\StringTok{ }
\StringTok{        }\KeywordTok{xlab}\NormalTok{(}\StringTok{"Weigth (Tonne)"}\NormalTok{) }\OperatorTok{+}\StringTok{ }\KeywordTok{ylab}\NormalTok{(}\StringTok{"Mile per Gallon"}\NormalTok{) }\OperatorTok{+}
\StringTok{        }\KeywordTok{geom_abline}\NormalTok{(}\KeywordTok{aes}\NormalTok{(}\DataTypeTok{intercept=}\NormalTok{intercept, }\DataTypeTok{slope=}\NormalTok{slope, }
                        \DataTypeTok{colour=}\KeywordTok{c}\NormalTok{(}\StringTok{"red"}\NormalTok{, }\StringTok{"blue"}\NormalTok{)), }\DataTypeTok{data=}\NormalTok{line) }\OperatorTok{+}\StringTok{ }
\StringTok{        }\KeywordTok{theme}\NormalTok{(}\DataTypeTok{legend.title=}\KeywordTok{element_blank}\NormalTok{()) }\OperatorTok{+}
\StringTok{        }\KeywordTok{scale_color_manual}\NormalTok{(}\DataTypeTok{labels =} \KeywordTok{c}\NormalTok{(}\StringTok{"manual"}\NormalTok{, }\StringTok{"automatic"}\NormalTok{), }\DataTypeTok{values =} \KeywordTok{c}\NormalTok{(}\StringTok{"blue"}\NormalTok{, }\StringTok{"red"}\NormalTok{)) }\OperatorTok{+}\StringTok{ }
\StringTok{        }\KeywordTok{geom_vline}\NormalTok{(}\DataTypeTok{xintercept =}\NormalTok{ x_com_wt, }\DataTypeTok{linetype =} \StringTok{"dashed"}\NormalTok{) }\OperatorTok{+}\StringTok{ }
\StringTok{        }\KeywordTok{geom_text}\NormalTok{(}\KeywordTok{aes}\NormalTok{(x_com_wt}\OperatorTok{+}\FloatTok{0.12}\NormalTok{,}\DecValTok{0}\NormalTok{,}\DataTypeTok{label =} \KeywordTok{round}\NormalTok{(x_com_wt, }\DecValTok{2}\NormalTok{)), }\DataTypeTok{size =} \DecValTok{3}\NormalTok{, }
                  \DataTypeTok{color =} \StringTok{"purple"}\NormalTok{)}
\end{Highlighting}
\end{Shaded}

The coefficients of fit\_disp:

\begin{Shaded}
\begin{Highlighting}[]
\NormalTok{cff_disp <-}\StringTok{ }\KeywordTok{round}\NormalTok{(}\KeywordTok{summary}\NormalTok{(fit_disp)}\OperatorTok{$}\NormalTok{coefficient, }\DecValTok{2}\NormalTok{)}
\NormalTok{cff_disp}
\end{Highlighting}
\end{Shaded}

The 95\% confidence interval for coefficients of fit\_disp:

\begin{Shaded}
\begin{Highlighting}[]
\NormalTok{q <-}\StringTok{ }\KeywordTok{c}\NormalTok{(}\OperatorTok{-}\DecValTok{1}\NormalTok{,}\DecValTok{1}\NormalTok{) }\OperatorTok{*}\StringTok{ }\KeywordTok{qt}\NormalTok{(.}\DecValTok{975}\NormalTok{, }\DataTypeTok{df =}\NormalTok{ fit_disp}\OperatorTok{$}\NormalTok{df)}
\NormalTok{int_aut <-}\StringTok{ }\NormalTok{cff_disp[}\DecValTok{1}\NormalTok{,}\DecValTok{1}\NormalTok{] }\OperatorTok{+}\StringTok{ }\NormalTok{q }\OperatorTok{*}\StringTok{ }\NormalTok{cff_disp[}\DecValTok{1}\NormalTok{,}\DecValTok{2}\NormalTok{] }
\NormalTok{int_aut2man <-}\StringTok{ }\NormalTok{cff_disp[}\DecValTok{3}\NormalTok{,}\DecValTok{1}\NormalTok{] }\OperatorTok{+}\StringTok{ }\NormalTok{q }\OperatorTok{*}\StringTok{ }\NormalTok{cff_disp[}\DecValTok{3}\NormalTok{,}\DecValTok{2}\NormalTok{] }
\NormalTok{disp_aut <-}\StringTok{ }\NormalTok{cff_disp[}\DecValTok{2}\NormalTok{,}\DecValTok{1}\NormalTok{] }\OperatorTok{+}\StringTok{ }\NormalTok{q }\OperatorTok{*}\StringTok{ }\NormalTok{cff_disp[}\DecValTok{2}\NormalTok{,}\DecValTok{2}\NormalTok{] }
\NormalTok{disp_aut2man <-}\StringTok{ }\NormalTok{cff_disp[}\DecValTok{4}\NormalTok{,}\DecValTok{1}\NormalTok{] }\OperatorTok{+}\StringTok{ }\NormalTok{q }\OperatorTok{*}\StringTok{ }\NormalTok{cff_disp[}\DecValTok{4}\NormalTok{,}\DecValTok{2}\NormalTok{] }
 

\NormalTok{conf_disp <-}\StringTok{ }\KeywordTok{data.frame}\NormalTok{(}\DataTypeTok{Intercept =}\NormalTok{ int_aut, }
                          \DataTypeTok{disp =}\NormalTok{ disp_aut,  }
                          \StringTok{`}\DataTypeTok{factor(am)manual}\StringTok{`}\NormalTok{ =}\StringTok{ }\NormalTok{int_aut2man, }
                          \StringTok{`}\DataTypeTok{disp:factor(am)manual}\StringTok{`}\NormalTok{ =}\StringTok{ }\NormalTok{disp_aut2man)}

\KeywordTok{row.names}\NormalTok{(conf_disp) <-}\StringTok{ }\KeywordTok{c}\NormalTok{(}\StringTok{"Lower"}\NormalTok{, }\StringTok{"Upper"}\NormalTok{)}

\KeywordTok{kable}\NormalTok{(}\KeywordTok{round}\NormalTok{(conf_disp, }\DecValTok{2}\NormalTok{), }\StringTok{"latex"}\NormalTok{, }\DataTypeTok{booktabs =}\NormalTok{ T) }\OperatorTok
\StringTok{  }\KeywordTok{kable_styling}\NormalTok{(}\DataTypeTok{position =} \StringTok{"center"}\NormalTok{)}
\end{Highlighting}
\end{Shaded}

The corresponding plot for fit\_disp:

\begin{Shaded}
\begin{Highlighting}[]
\NormalTok{line <-}\StringTok{ }\KeywordTok{data.frame}\NormalTok{(}\DataTypeTok{intercept =} \KeywordTok{c}\NormalTok{( cff_disp[}\DecValTok{1}\NormalTok{, }\DecValTok{1}\NormalTok{], cff_disp[}\DecValTok{1}\NormalTok{, }\DecValTok{1}\NormalTok{] }\OperatorTok{+}\StringTok{ }\NormalTok{cff_disp[}\DecValTok{3}\NormalTok{, }\DecValTok{1}\NormalTok{]),}
                   \DataTypeTok{slope =} \KeywordTok{c}\NormalTok{( cff_disp[}\DecValTok{2}\NormalTok{, }\DecValTok{1}\NormalTok{], cff_disp[}\DecValTok{2}\NormalTok{, }\DecValTok{1}\NormalTok{] }\OperatorTok{+}\StringTok{ }\NormalTok{cff_disp[}\DecValTok{4}\NormalTok{, }\DecValTok{1}\NormalTok{]), }
                   \DataTypeTok{row.names =} \KeywordTok{c}\NormalTok{(}\StringTok{"automatic"}\NormalTok{, }\StringTok{"manual"}\NormalTok{) )}
\NormalTok{x_com_disp <-}\StringTok{ }\NormalTok{(line[}\DecValTok{1}\NormalTok{,}\DecValTok{1}\NormalTok{] }\OperatorTok{-}\StringTok{ }\NormalTok{line[}\DecValTok{2}\NormalTok{,}\DecValTok{1}\NormalTok{]) }\OperatorTok{/}\StringTok{ }\NormalTok{(line[}\DecValTok{2}\NormalTok{,}\DecValTok{2}\NormalTok{] }\OperatorTok{-}\StringTok{ }\NormalTok{line[}\DecValTok{1}\NormalTok{, }\DecValTok{2}\NormalTok{]) }
\KeywordTok{qplot}\NormalTok{(disp, mpg, }\DataTypeTok{data =}\NormalTok{ mtcars) }\OperatorTok{+}
\StringTok{        }\KeywordTok{geom_abline}\NormalTok{(}\KeywordTok{aes}\NormalTok{(}\DataTypeTok{intercept=}\NormalTok{intercept, }\DataTypeTok{slope=}\NormalTok{slope,}
                        \DataTypeTok{colour=}\KeywordTok{c}\NormalTok{(}\StringTok{"blue"}\NormalTok{, }\StringTok{"red"}\NormalTok{)), }\DataTypeTok{data=}\NormalTok{line) }\OperatorTok{+}\StringTok{ }
\StringTok{        }\KeywordTok{theme}\NormalTok{(}\DataTypeTok{legend.title=}\KeywordTok{element_blank}\NormalTok{()) }\OperatorTok{+}
\StringTok{        }\KeywordTok{scale_color_manual}\NormalTok{(}\DataTypeTok{labels =} \KeywordTok{c}\NormalTok{(}\StringTok{"automatic"}\NormalTok{, }\StringTok{"manual"}\NormalTok{), }\DataTypeTok{values =} \KeywordTok{c}\NormalTok{(}\StringTok{"red"}\NormalTok{, }\StringTok{"blue"}\NormalTok{)) }\OperatorTok{+}\StringTok{ }
\StringTok{        }\KeywordTok{geom_vline}\NormalTok{(}\DataTypeTok{xintercept =}\NormalTok{ x_com_disp, }\DataTypeTok{linetype =} \StringTok{"dashed"}\NormalTok{) }\OperatorTok{+}\StringTok{ }
\StringTok{        }\KeywordTok{geom_text}\NormalTok{(}\KeywordTok{aes}\NormalTok{(x_com_disp}\OperatorTok{+}\DecValTok{10}\NormalTok{,}\DecValTok{0}\NormalTok{,}\DataTypeTok{label =} \KeywordTok{round}\NormalTok{(x_com_disp, }\DecValTok{2}\NormalTok{)), }\DataTypeTok{size =} \DecValTok{3}\NormalTok{,}
                  \DataTypeTok{color =} \StringTok{"purple"}\NormalTok{)}
\end{Highlighting}
\end{Shaded}

The coefficients of fit\_hp:

\begin{Shaded}
\begin{Highlighting}[]
\NormalTok{cff_hp <-}\StringTok{ }\KeywordTok{round}\NormalTok{(}\KeywordTok{summary}\NormalTok{(fit_hp)}\OperatorTok{$}\NormalTok{coefficient, }\DecValTok{2}\NormalTok{)}
\NormalTok{cff_hp}
\end{Highlighting}
\end{Shaded}

The 95\% confidence interval for the coefficients of fit\_hp:

\begin{Shaded}
\begin{Highlighting}[]
\NormalTok{q <-}\StringTok{ }\KeywordTok{c}\NormalTok{(}\OperatorTok{-}\DecValTok{1}\NormalTok{,}\DecValTok{1}\NormalTok{) }\OperatorTok{*}\StringTok{ }\KeywordTok{qt}\NormalTok{(.}\DecValTok{975}\NormalTok{, }\DataTypeTok{df =}\NormalTok{ fit_hp}\OperatorTok{$}\NormalTok{df)}
\NormalTok{int_aut <-}\StringTok{ }\NormalTok{cff_hp[}\DecValTok{1}\NormalTok{,}\DecValTok{1}\NormalTok{] }\OperatorTok{+}\StringTok{ }\NormalTok{q }\OperatorTok{*}\StringTok{ }\NormalTok{cff_hp[}\DecValTok{1}\NormalTok{,}\DecValTok{2}\NormalTok{] }
\NormalTok{int_aut2man <-}\StringTok{ }\NormalTok{cff_hp[}\DecValTok{3}\NormalTok{,}\DecValTok{1}\NormalTok{] }\OperatorTok{+}\StringTok{ }\NormalTok{q }\OperatorTok{*}\StringTok{ }\NormalTok{cff_hp[}\DecValTok{3}\NormalTok{,}\DecValTok{2}\NormalTok{] }
\NormalTok{hp_aut <-}\StringTok{ }\NormalTok{cff_hp[}\DecValTok{2}\NormalTok{,}\DecValTok{1}\NormalTok{] }\OperatorTok{+}\StringTok{ }\NormalTok{q }\OperatorTok{*}\StringTok{ }\NormalTok{cff_hp[}\DecValTok{2}\NormalTok{,}\DecValTok{2}\NormalTok{] }
\NormalTok{hp_aut2man <-}\StringTok{ }\NormalTok{cff_hp[}\DecValTok{4}\NormalTok{,}\DecValTok{1}\NormalTok{] }\OperatorTok{+}\StringTok{ }\NormalTok{q }\OperatorTok{*}\StringTok{ }\NormalTok{cff_hp[}\DecValTok{4}\NormalTok{,}\DecValTok{2}\NormalTok{] }
 

\NormalTok{conf_hp <-}\StringTok{ }\KeywordTok{data.frame}\NormalTok{(}\DataTypeTok{Intercept =}\NormalTok{ int_aut, }
                          \DataTypeTok{hp =}\NormalTok{ hp_aut,  }
                          \StringTok{`}\DataTypeTok{factor(am)manual}\StringTok{`}\NormalTok{ =}\StringTok{ }\NormalTok{int_aut2man, }
                          \StringTok{`}\DataTypeTok{hp:factor(am)manual}\StringTok{`}\NormalTok{ =}\StringTok{ }\NormalTok{hp_aut2man)}

\KeywordTok{row.names}\NormalTok{(conf_hp) <-}\StringTok{ }\KeywordTok{c}\NormalTok{(}\StringTok{"Lower"}\NormalTok{, }\StringTok{"Upper"}\NormalTok{)}

\KeywordTok{kable}\NormalTok{(}\KeywordTok{round}\NormalTok{(conf_hp, }\DecValTok{2}\NormalTok{), }\StringTok{"latex"}\NormalTok{, }\DataTypeTok{booktabs =}\NormalTok{ T) }\OperatorTok
\StringTok{  }\KeywordTok{kable_styling}\NormalTok{(}\DataTypeTok{position =} \StringTok{"center"}\NormalTok{)}
\end{Highlighting}
\end{Shaded}

The corresponding plot for fit\_hp:

\begin{Shaded}
\begin{Highlighting}[]
\NormalTok{line <-}\StringTok{ }\KeywordTok{data.frame}\NormalTok{(}\DataTypeTok{intercept =} \KeywordTok{c}\NormalTok{( cff_hp[}\DecValTok{1}\NormalTok{, }\DecValTok{1}\NormalTok{], cff_hp[}\DecValTok{1}\NormalTok{, }\DecValTok{1}\NormalTok{] }\OperatorTok{+}\StringTok{ }\NormalTok{cff_hp[}\DecValTok{3}\NormalTok{, }\DecValTok{1}\NormalTok{]),}
                   \DataTypeTok{slope =} \KeywordTok{c}\NormalTok{( cff_hp[}\DecValTok{2}\NormalTok{, }\DecValTok{1}\NormalTok{], cff_hp[}\DecValTok{2}\NormalTok{, }\DecValTok{1}\NormalTok{] }\OperatorTok{+}\StringTok{ }\NormalTok{cff_hp[}\DecValTok{4}\NormalTok{, }\DecValTok{1}\NormalTok{]), }
                   \DataTypeTok{row.names =} \KeywordTok{c}\NormalTok{(}\StringTok{"automatic"}\NormalTok{, }\StringTok{"manual"}\NormalTok{) )}

\KeywordTok{qplot}\NormalTok{(hp, mpg, }\DataTypeTok{data =}\NormalTok{ mtcars) }\OperatorTok{+}
\StringTok{        }\KeywordTok{geom_abline}\NormalTok{(}\KeywordTok{aes}\NormalTok{(}\DataTypeTok{intercept=}\NormalTok{intercept, }\DataTypeTok{slope=}\NormalTok{slope,}
                        \DataTypeTok{colour=}\KeywordTok{c}\NormalTok{(}\StringTok{"blue"}\NormalTok{, }\StringTok{"red"}\NormalTok{)), }\DataTypeTok{data=}\NormalTok{line) }\OperatorTok{+}\StringTok{ }
\StringTok{        }\KeywordTok{theme}\NormalTok{(}\DataTypeTok{legend.title=}\KeywordTok{element_blank}\NormalTok{()) }\OperatorTok{+}
\StringTok{        }\KeywordTok{scale_color_manual}\NormalTok{(}\DataTypeTok{labels =} \KeywordTok{c}\NormalTok{(}\StringTok{"automatic"}\NormalTok{, }\StringTok{"manual"}\NormalTok{), }\DataTypeTok{values =} \KeywordTok{c}\NormalTok{(}\StringTok{"red"}\NormalTok{, }\StringTok{"blue"}\NormalTok{))}
\end{Highlighting}
\end{Shaded}

\hypertarget{app-code-4}{\section*{\texorpdfstring{Appendix C: R Scripts
of \protect\hyperlink{sec-mult}{Sect.
4}}{Appendix C: R Scripts of Sect. 4}}\label{app-code-4}}
\addcontentsline{toc}{section}{Appendix C: R Scripts of
\protect\hyperlink{sec-mult}{Sect. 4}}

The script for Table. \ref{tab:test-mult-wt}

\begin{Shaded}
\begin{Highlighting}[]
\NormalTok{fit11 <-}\StringTok{ }\KeywordTok{lm}\NormalTok{(mpg }\OperatorTok{~}\StringTok{ }\NormalTok{wt, }\DataTypeTok{data =}\NormalTok{ mtcars)}
\NormalTok{fit12 <-}\StringTok{ }\KeywordTok{lm}\NormalTok{(mpg }\OperatorTok{~}\StringTok{ }\NormalTok{wt }\OperatorTok{+}\StringTok{ }\NormalTok{disp, }\DataTypeTok{data =}\NormalTok{ mtcars)}
\NormalTok{fit13 <-}\StringTok{ }\KeywordTok{lm}\NormalTok{(mpg }\OperatorTok{~}\StringTok{ }\NormalTok{wt }\OperatorTok{+}\StringTok{ }\NormalTok{hp, }\DataTypeTok{data =}\NormalTok{ mtcars)}
\NormalTok{fit14 <-}\StringTok{ }\KeywordTok{lm}\NormalTok{(mpg }\OperatorTok{~}\StringTok{ }\NormalTok{wt }\OperatorTok{+}\StringTok{ }\NormalTok{drat, }\DataTypeTok{data =}\NormalTok{ mtcars)}
\NormalTok{fit15 <-}\StringTok{ }\KeywordTok{lm}\NormalTok{(mpg }\OperatorTok{~}\StringTok{ }\NormalTok{wt }\OperatorTok{+}\StringTok{ }\NormalTok{qsec, }\DataTypeTok{data =}\NormalTok{ mtcars)}

\NormalTok{disp_val <-}\StringTok{ }\KeywordTok{round}\NormalTok{(}\KeywordTok{anova}\NormalTok{(fit11, fit12)}\OperatorTok{$}\StringTok{`}\DataTypeTok{Pr(>F)}\StringTok{`}\NormalTok{[}\DecValTok{2}\NormalTok{], }\DecValTok{2}\NormalTok{)}
\NormalTok{hp_val <-}\StringTok{ }\KeywordTok{round}\NormalTok{(}\KeywordTok{anova}\NormalTok{(fit11, fit13)}\OperatorTok{$}\StringTok{`}\DataTypeTok{Pr(>F)}\StringTok{`}\NormalTok{[}\DecValTok{2}\NormalTok{], }\DecValTok{2}\NormalTok{)}
\NormalTok{drat_val <-}\StringTok{ }\KeywordTok{round}\NormalTok{(}\KeywordTok{anova}\NormalTok{(fit11, fit14)}\OperatorTok{$}\StringTok{`}\DataTypeTok{Pr(>F)}\StringTok{`}\NormalTok{[}\DecValTok{2}\NormalTok{], }\DecValTok{2}\NormalTok{)}
\NormalTok{qsec_val <-}\StringTok{ }\KeywordTok{round}\NormalTok{(}\KeywordTok{anova}\NormalTok{(fit11, fit15)}\OperatorTok{$}\StringTok{`}\DataTypeTok{Pr(>F)}\StringTok{`}\NormalTok{[}\DecValTok{2}\NormalTok{], }\DecValTok{2}\NormalTok{)}


\NormalTok{tests_pval <-}\StringTok{ }\KeywordTok{as.data.frame}\NormalTok{(}\KeywordTok{cbind}\NormalTok{(disp_val, hp_val, drat_val, qsec_val))}
\KeywordTok{row.names}\NormalTok{(tests_pval) <-}\StringTok{ }\KeywordTok{c}\NormalTok{(}\StringTok{"mpg~wt"}\NormalTok{)}
\KeywordTok{colnames}\NormalTok{(tests_pval) <-}\StringTok{ }\KeywordTok{c}\NormalTok{(}\StringTok{"mpg~wt+disp"}\NormalTok{, }\StringTok{"  mpg~wt+hp"}\NormalTok{, }\StringTok{"  mpg~wt+drat"}\NormalTok{, }\StringTok{"  mpg~wt+qsec"}\NormalTok{)}

\KeywordTok{kable}\NormalTok{(tests_pval, }\StringTok{"latex"}\NormalTok{, }\DataTypeTok{booktabs =}\NormalTok{ T, }
      \DataTypeTok{caption =} \StringTok{"P-values of Comparing 2-variate Models with wt as the Regressor"}\NormalTok{) }\OperatorTok
\StringTok{        }\KeywordTok{kable_styling}\NormalTok{(}\DataTypeTok{latex_options =} \StringTok{"hold_position"}\NormalTok{)}
\end{Highlighting}
\end{Shaded}

Comparing two models mpg\textasciitilde{}wt+hp and
mpg\textasciitilde{}wt+qsec:

\begin{Shaded}
\begin{Highlighting}[]
\KeywordTok{round}\NormalTok{(}\KeywordTok{coxtest}\NormalTok{(fit13, fit15), }\DecValTok{2}\NormalTok{)}
\end{Highlighting}
\end{Shaded}

Comparing two models mpg\textasciitilde{}wt+hp+qsec and
mpg\textasciitilde{}wt+hp:

\begin{Shaded}
\begin{Highlighting}[]
\NormalTok{fit <-}\StringTok{ }\KeywordTok{lm}\NormalTok{(mpg }\OperatorTok{~}\StringTok{ }\NormalTok{wt }\OperatorTok{+}\StringTok{ }\NormalTok{hp }\OperatorTok{+}\StringTok{ }\NormalTok{qsec, }\DataTypeTok{data =}\NormalTok{ mtcars)}
\KeywordTok{anova}\NormalTok{(fit13, fit)}
\end{Highlighting}
\end{Shaded}

Comparing two models mpg\textasciitilde{}wt+hp+qsec and
mpg\textasciitilde{}wt+qsec:

\begin{Shaded}
\begin{Highlighting}[]
\KeywordTok{anova}\NormalTok{(fit15, fit)}
\end{Highlighting}
\end{Shaded}

The script fot Table. \ref{tab:test-mult-hp}:

\begin{Shaded}
\begin{Highlighting}[]
\NormalTok{fit21 <-}\StringTok{ }\KeywordTok{lm}\NormalTok{(mpg }\OperatorTok{~}\StringTok{ }\NormalTok{hp, }\DataTypeTok{data =}\NormalTok{ mtcars)}
\NormalTok{fit22 <-}\StringTok{ }\KeywordTok{lm}\NormalTok{(mpg }\OperatorTok{~}\StringTok{ }\NormalTok{hp }\OperatorTok{+}\StringTok{ }\NormalTok{disp, }\DataTypeTok{data =}\NormalTok{ mtcars)}
\NormalTok{fit23 <-}\StringTok{ }\KeywordTok{lm}\NormalTok{(mpg }\OperatorTok{~}\StringTok{ }\NormalTok{hp }\OperatorTok{+}\StringTok{ }\NormalTok{wt, }\DataTypeTok{data =}\NormalTok{ mtcars)}
\NormalTok{fit24 <-}\StringTok{ }\KeywordTok{lm}\NormalTok{(mpg }\OperatorTok{~}\StringTok{ }\NormalTok{hp }\OperatorTok{+}\StringTok{ }\NormalTok{drat, }\DataTypeTok{data =}\NormalTok{ mtcars)}
\NormalTok{fit25 <-}\StringTok{ }\KeywordTok{lm}\NormalTok{(mpg }\OperatorTok{~}\StringTok{ }\NormalTok{hp }\OperatorTok{+}\StringTok{ }\NormalTok{qsec, }\DataTypeTok{data =}\NormalTok{ mtcars)}

\NormalTok{disp_val <-}\StringTok{ }\KeywordTok{round}\NormalTok{(}\KeywordTok{anova}\NormalTok{(fit21, fit22)}\OperatorTok{$}\StringTok{`}\DataTypeTok{Pr(>F)}\StringTok{`}\NormalTok{[}\DecValTok{2}\NormalTok{], }\DecValTok{2}\NormalTok{)}
\NormalTok{wt_val <-}\StringTok{ }\KeywordTok{round}\NormalTok{(}\KeywordTok{anova}\NormalTok{(fit21, fit23)}\OperatorTok{$}\StringTok{`}\DataTypeTok{Pr(>F)}\StringTok{`}\NormalTok{[}\DecValTok{2}\NormalTok{], }\DecValTok{2}\NormalTok{)}
\NormalTok{drat_val <-}\StringTok{ }\KeywordTok{round}\NormalTok{(}\KeywordTok{anova}\NormalTok{(fit21, fit24)}\OperatorTok{$}\StringTok{`}\DataTypeTok{Pr(>F)}\StringTok{`}\NormalTok{[}\DecValTok{2}\NormalTok{], }\DecValTok{2}\NormalTok{)}
\NormalTok{qsec_val <-}\StringTok{ }\KeywordTok{round}\NormalTok{(}\KeywordTok{anova}\NormalTok{(fit21, fit25)}\OperatorTok{$}\StringTok{`}\DataTypeTok{Pr(>F)}\StringTok{`}\NormalTok{[}\DecValTok{2}\NormalTok{], }\DecValTok{2}\NormalTok{)}


\NormalTok{tests_pval <-}\StringTok{ }\KeywordTok{as.data.frame}\NormalTok{(}\KeywordTok{cbind}\NormalTok{(disp_val, wt_val, drat_val, qsec_val))}
\KeywordTok{row.names}\NormalTok{(tests_pval) <-}\StringTok{ }\KeywordTok{c}\NormalTok{(}\StringTok{"mpg~hp"}\NormalTok{)}
\KeywordTok{colnames}\NormalTok{(tests_pval) <-}\StringTok{ }\KeywordTok{c}\NormalTok{(}\StringTok{"mpg~hp+disp"}\NormalTok{, }\StringTok{"  mpg~hp+wt"}\NormalTok{, }\StringTok{"  mpg~hp+drat"}\NormalTok{, }\StringTok{"  mpg~hp+qsec"}\NormalTok{)}

\KeywordTok{kable}\NormalTok{(tests_pval, }\StringTok{"latex"}\NormalTok{, }\DataTypeTok{booktabs =}\NormalTok{ T, }
      \DataTypeTok{caption =} \StringTok{"P-values of Comparing 2-variate Models with hp as the Regressor"}\NormalTok{) }\OperatorTok
\StringTok{        }\KeywordTok{kable_styling}\NormalTok{(}\DataTypeTok{latex_options =} \StringTok{"hold_position"}\NormalTok{)}
\end{Highlighting}
\end{Shaded}

Comparing mpg\textasciitilde{}hp+wt vs.~mpg\textasciitilde{}hp+disp and
mpg\textasciitilde{}hp+drat:

\begin{Shaded}
\begin{Highlighting}[]
\KeywordTok{coxtest}\NormalTok{(fit22, fit23)}
\KeywordTok{coxtest}\NormalTok{(fit23, fit24)}
\end{Highlighting}
\end{Shaded}

The script for Table. \ref{tab:test-mult-disp}.

\begin{Shaded}
\begin{Highlighting}[]
\NormalTok{fit31 <-}\StringTok{ }\KeywordTok{lm}\NormalTok{(mpg }\OperatorTok{~}\StringTok{ }\NormalTok{disp, }\DataTypeTok{data =}\NormalTok{ mtcars)}
\NormalTok{fit32 <-}\StringTok{ }\KeywordTok{lm}\NormalTok{(mpg }\OperatorTok{~}\StringTok{ }\NormalTok{disp }\OperatorTok{+}\StringTok{ }\NormalTok{hp, }\DataTypeTok{data =}\NormalTok{ mtcars)}
\NormalTok{fit33 <-}\StringTok{ }\KeywordTok{lm}\NormalTok{(mpg }\OperatorTok{~}\StringTok{ }\NormalTok{disp }\OperatorTok{+}\StringTok{ }\NormalTok{wt, }\DataTypeTok{data =}\NormalTok{ mtcars)}
\NormalTok{fit34 <-}\StringTok{ }\KeywordTok{lm}\NormalTok{(mpg }\OperatorTok{~}\StringTok{ }\NormalTok{disp }\OperatorTok{+}\StringTok{ }\NormalTok{drat, }\DataTypeTok{data =}\NormalTok{ mtcars)}
\NormalTok{fit35 <-}\StringTok{ }\KeywordTok{lm}\NormalTok{(mpg }\OperatorTok{~}\StringTok{ }\NormalTok{disp }\OperatorTok{+}\StringTok{ }\NormalTok{qsec, }\DataTypeTok{data =}\NormalTok{ mtcars)}

\NormalTok{hp_val <-}\StringTok{ }\KeywordTok{round}\NormalTok{(}\KeywordTok{anova}\NormalTok{(fit31, fit32)}\OperatorTok{$}\StringTok{`}\DataTypeTok{Pr(>F)}\StringTok{`}\NormalTok{[}\DecValTok{2}\NormalTok{], }\DecValTok{2}\NormalTok{)}
\NormalTok{wt_val <-}\StringTok{ }\KeywordTok{round}\NormalTok{(}\KeywordTok{anova}\NormalTok{(fit31, fit33)}\OperatorTok{$}\StringTok{`}\DataTypeTok{Pr(>F)}\StringTok{`}\NormalTok{[}\DecValTok{2}\NormalTok{], }\DecValTok{2}\NormalTok{)}
\NormalTok{drat_val <-}\StringTok{ }\KeywordTok{round}\NormalTok{(}\KeywordTok{anova}\NormalTok{(fit31, fit34)}\OperatorTok{$}\StringTok{`}\DataTypeTok{Pr(>F)}\StringTok{`}\NormalTok{[}\DecValTok{2}\NormalTok{], }\DecValTok{2}\NormalTok{)}
\NormalTok{qsec_val <-}\StringTok{ }\KeywordTok{round}\NormalTok{(}\KeywordTok{anova}\NormalTok{(fit31, fit35)}\OperatorTok{$}\StringTok{`}\DataTypeTok{Pr(>F)}\StringTok{`}\NormalTok{[}\DecValTok{2}\NormalTok{], }\DecValTok{2}\NormalTok{)}


\NormalTok{tests_pval <-}\StringTok{ }\KeywordTok{as.data.frame}\NormalTok{(}\KeywordTok{cbind}\NormalTok{(hp_val, wt_val, drat_val, qsec_val))}
\KeywordTok{row.names}\NormalTok{(tests_pval) <-}\StringTok{ }\KeywordTok{c}\NormalTok{(}\StringTok{"mpg~disp"}\NormalTok{)}
\KeywordTok{colnames}\NormalTok{(tests_pval) <-}\StringTok{ }
\StringTok{        }\KeywordTok{c}\NormalTok{(}\StringTok{"mpg~disp+hp"}\NormalTok{, }\StringTok{"  mpg~disp+wt"}\NormalTok{, }\StringTok{"  mpg~disp+drat"}\NormalTok{, }\StringTok{"  mpg~dispp+qsec"}\NormalTok{)}

\KeywordTok{kable}\NormalTok{(tests_pval, }\StringTok{"latex"}\NormalTok{, }\DataTypeTok{booktabs =}\NormalTok{ T,}
      \DataTypeTok{caption =} \StringTok{"P-values of Comparing 2-variate Models with disp as the Regressor"}\NormalTok{) }\OperatorTok
\StringTok{        }\KeywordTok{kable_styling}\NormalTok{(}\DataTypeTok{latex_options =} \StringTok{"hold_position"}\NormalTok{)}
\end{Highlighting}
\end{Shaded}

The coefficients of the fit\_wt\_hp:

\begin{Shaded}
\begin{Highlighting}[]
\NormalTok{cff_wt_hp <-}\StringTok{ }\KeywordTok{round}\NormalTok{(}\KeywordTok{summary}\NormalTok{(fit_wt_hp)}\OperatorTok{$}\NormalTok{coefficient, }\DecValTok{2}\NormalTok{)}
\NormalTok{cff_wt_hp}
\end{Highlighting}
\end{Shaded}

The coefficients of the new fit\_wt\_hp:

\begin{Shaded}
\begin{Highlighting}[]
\NormalTok{cff_wt_hp <-}\StringTok{ }\KeywordTok{round}\NormalTok{(}\KeywordTok{summary}\NormalTok{(fit_wt_hp)}\OperatorTok{$}\NormalTok{coefficient, }\DecValTok{2}\NormalTok{)}
\NormalTok{cff_wt_hp}
\end{Highlighting}
\end{Shaded}

The confidence intervals for coefficients for fit\_wt\_hp:

\begin{Shaded}
\begin{Highlighting}[]
\NormalTok{q <-}\StringTok{ }\KeywordTok{c}\NormalTok{(}\OperatorTok{-}\DecValTok{1}\NormalTok{,}\DecValTok{1}\NormalTok{) }\OperatorTok{*}\StringTok{ }\KeywordTok{qt}\NormalTok{(.}\DecValTok{975}\NormalTok{, }\DataTypeTok{df =}\NormalTok{ fit_wt_hp}\OperatorTok{$}\NormalTok{df)}
\NormalTok{int_aut <-}\StringTok{ }\NormalTok{cff_wt_hp[}\DecValTok{1}\NormalTok{,}\DecValTok{1}\NormalTok{] }\OperatorTok{+}\StringTok{ }\NormalTok{q }\OperatorTok{*}\StringTok{ }\NormalTok{cff_wt_hp[}\DecValTok{1}\NormalTok{,}\DecValTok{2}\NormalTok{] }
\NormalTok{int_aut2man <-}\StringTok{ }\NormalTok{cff_wt_hp[}\DecValTok{3}\NormalTok{,}\DecValTok{1}\NormalTok{] }\OperatorTok{+}\StringTok{ }\NormalTok{q }\OperatorTok{*}\StringTok{ }\NormalTok{cff_wt_hp[}\DecValTok{3}\NormalTok{,}\DecValTok{2}\NormalTok{] }
\NormalTok{wt_aut <-}\StringTok{ }\NormalTok{cff_wt_hp[}\DecValTok{2}\NormalTok{,}\DecValTok{1}\NormalTok{] }\OperatorTok{+}\StringTok{ }\NormalTok{q }\OperatorTok{*}\StringTok{ }\NormalTok{cff_wt_hp[}\DecValTok{2}\NormalTok{,}\DecValTok{2}\NormalTok{] }
\NormalTok{wt_aut2man <-}\StringTok{ }\NormalTok{cff_wt_hp[}\DecValTok{5}\NormalTok{,}\DecValTok{1}\NormalTok{] }\OperatorTok{+}\StringTok{ }\NormalTok{q }\OperatorTok{*}\StringTok{ }\NormalTok{cff_wt_hp[}\DecValTok{5}\NormalTok{,}\DecValTok{2}\NormalTok{] }
\NormalTok{hpc <-}\StringTok{ }\NormalTok{cff_wt_hp[}\DecValTok{4}\NormalTok{,}\DecValTok{1}\NormalTok{] }\OperatorTok{+}\StringTok{ }\NormalTok{q }\OperatorTok{*}\StringTok{ }\NormalTok{cff_wt_hp[}\DecValTok{4}\NormalTok{,}\DecValTok{2}\NormalTok{] }

\NormalTok{conf_wthp <-}\StringTok{ }\KeywordTok{data.frame}\NormalTok{(}\DataTypeTok{Intercept =}\NormalTok{ int_aut, }
                        \DataTypeTok{wt =}\NormalTok{ wt_aut,  }
                        \StringTok{`}\DataTypeTok{factor(am)manual}\StringTok{`}\NormalTok{ =}\StringTok{ }\NormalTok{int_aut2man, }
                        \DataTypeTok{hp =}\NormalTok{ hpc, }
                        \StringTok{`}\DataTypeTok{wt:factor(am)manual}\StringTok{`}\NormalTok{ =}\StringTok{ }\NormalTok{wt_aut2man)}

\KeywordTok{row.names}\NormalTok{(conf_wthp) <-}\StringTok{ }\KeywordTok{c}\NormalTok{(}\StringTok{"Lower"}\NormalTok{, }\StringTok{"Upper"}\NormalTok{)}

\KeywordTok{kable}\NormalTok{(}\KeywordTok{round}\NormalTok{(conf_wthp, }\DecValTok{2}\NormalTok{), }\StringTok{"latex"}\NormalTok{, }\DataTypeTok{booktabs =}\NormalTok{ T) }\OperatorTok
\StringTok{  }\KeywordTok{kable_styling}\NormalTok{(}\DataTypeTok{position =} \StringTok{"center"}\NormalTok{)}
\end{Highlighting}
\end{Shaded}

The coefficients of the model fit\_wt\_qsec:

\begin{Shaded}
\begin{Highlighting}[]
\NormalTok{cff_wt_qsec <-}\StringTok{ }\KeywordTok{round}\NormalTok{(}\KeywordTok{summary}\NormalTok{(fit_wt_qsec)}\OperatorTok{$}\NormalTok{coefficient, }\DecValTok{2}\NormalTok{)}
\NormalTok{cff_wt_qsec}
\end{Highlighting}
\end{Shaded}

The coefficients of the new model fit\_wt\_qsec:

\begin{Shaded}
\begin{Highlighting}[]
\NormalTok{cff_wt_qsec <-}\StringTok{ }\KeywordTok{round}\NormalTok{(}\KeywordTok{summary}\NormalTok{(fit_wt_qsec)}\OperatorTok{$}\NormalTok{coefficient, }\DecValTok{2}\NormalTok{)}
\NormalTok{cff_wt_qsec}
\end{Highlighting}
\end{Shaded}

The confidence intervals for coefficients for fit\_wt\_qsec:

\begin{Shaded}
\begin{Highlighting}[]
\NormalTok{q <-}\StringTok{ }\KeywordTok{c}\NormalTok{(}\OperatorTok{-}\DecValTok{1}\NormalTok{,}\DecValTok{1}\NormalTok{) }\OperatorTok{*}\StringTok{ }\KeywordTok{qt}\NormalTok{(.}\DecValTok{975}\NormalTok{, }\DataTypeTok{df =}\NormalTok{ fit_wt_qsec}\OperatorTok{$}\NormalTok{df)}
\NormalTok{int_aut <-}\StringTok{ }\NormalTok{cff_wt_qsec[}\DecValTok{1}\NormalTok{,}\DecValTok{1}\NormalTok{] }\OperatorTok{+}\StringTok{ }\NormalTok{q }\OperatorTok{*}\StringTok{ }\NormalTok{cff_wt_qsec[}\DecValTok{1}\NormalTok{,}\DecValTok{2}\NormalTok{] }
\NormalTok{int_aut2man <-}\StringTok{ }\NormalTok{cff_wt_qsec[}\DecValTok{3}\NormalTok{,}\DecValTok{1}\NormalTok{] }\OperatorTok{+}\StringTok{ }\NormalTok{q }\OperatorTok{*}\StringTok{ }\NormalTok{cff_wt_qsec[}\DecValTok{3}\NormalTok{,}\DecValTok{2}\NormalTok{] }
\NormalTok{wt_aut <-}\StringTok{ }\NormalTok{cff_wt_qsec[}\DecValTok{2}\NormalTok{,}\DecValTok{1}\NormalTok{] }\OperatorTok{+}\StringTok{ }\NormalTok{q }\OperatorTok{*}\StringTok{ }\NormalTok{cff_wt_qsec[}\DecValTok{2}\NormalTok{,}\DecValTok{2}\NormalTok{] }
\NormalTok{wt_aut2man <-}\StringTok{ }\NormalTok{cff_wt_qsec[}\DecValTok{5}\NormalTok{,}\DecValTok{1}\NormalTok{] }\OperatorTok{+}\StringTok{ }\NormalTok{q }\OperatorTok{*}\StringTok{ }\NormalTok{cff_wt_qsec[}\DecValTok{5}\NormalTok{,}\DecValTok{2}\NormalTok{] }
\NormalTok{qsecc <-}\StringTok{ }\NormalTok{cff_wt_qsec[}\DecValTok{4}\NormalTok{,}\DecValTok{1}\NormalTok{] }\OperatorTok{+}\StringTok{ }\NormalTok{q }\OperatorTok{*}\StringTok{ }\NormalTok{cff_wt_qsec[}\DecValTok{4}\NormalTok{,}\DecValTok{2}\NormalTok{] }

\NormalTok{conf_wtqsec <-}\StringTok{ }\KeywordTok{data.frame}\NormalTok{(}\DataTypeTok{Intercept =}\NormalTok{ int_aut, }
                          \DataTypeTok{wt =}\NormalTok{ wt_aut,  }
                          \StringTok{`}\DataTypeTok{factor(am)manual}\StringTok{`}\NormalTok{ =}\StringTok{ }\NormalTok{int_aut2man, }
                          \DataTypeTok{qsec =}\NormalTok{ qsecc, }
                          \StringTok{`}\DataTypeTok{wt:factor(am)manual}\StringTok{`}\NormalTok{ =}\StringTok{ }\NormalTok{wt_aut2man)}

\KeywordTok{row.names}\NormalTok{(conf_wtqsec) <-}\StringTok{ }\KeywordTok{c}\NormalTok{(}\StringTok{"Lower"}\NormalTok{, }\StringTok{"Upper"}\NormalTok{)}

\KeywordTok{kable}\NormalTok{(}\KeywordTok{round}\NormalTok{(conf_wtqsec, }\DecValTok{2}\NormalTok{), }\StringTok{"latex"}\NormalTok{, }\DataTypeTok{booktabs =}\NormalTok{ T) }\OperatorTok
\StringTok{  }\KeywordTok{kable_styling}\NormalTok{(}\DataTypeTok{position =} \StringTok{"center"}\NormalTok{)}
\end{Highlighting}
\end{Shaded}

\hypertarget{app-diag}{\section*{Appendix D: The Diagnosis
Plots}\label{app-diag}}
\addcontentsline{toc}{section}{Appendix D: The Diagnosis Plots}

\begin{figure}[H]

{\centering \includegraphics{D7_MPG_files/figure-latex/unnamed-chunk-30-1} 

}

\caption{\label{fig:diag-1-wt}Diagnosis Plots for mpg vs. wt*factor(am)}\label{fig:unnamed-chunk-30}
\end{figure}

\begin{figure}[H]

{\centering \includegraphics{D7_MPG_files/figure-latex/unnamed-chunk-31-1} 

}

\caption{\label{fig:diag-1-disp}Diagnosis Plots for mpg vs. disp*factor(am)}\label{fig:unnamed-chunk-31}
\end{figure}

\begin{figure}[H]

{\centering \includegraphics{D7_MPG_files/figure-latex/unnamed-chunk-32-1} 

}

\caption{\label{fig:diag-1-hp}Diagnosis Plots for mpg vs. hp*factor(am)}\label{fig:unnamed-chunk-32}
\end{figure}

\begin{figure}[H]

{\centering \includegraphics{D7_MPG_files/figure-latex/unnamed-chunk-33-1} 

}

\caption{\label{fig:diag-2-hp}Diagnosis Plots for mpg vs. (wt + hp) * factor(am)}\label{fig:unnamed-chunk-33}
\end{figure}

\begin{figure}[H]

{\centering \includegraphics{D7_MPG_files/figure-latex/unnamed-chunk-34-1} 

}

\caption{\label{fig:diag-2-qsec}Diagnosis Plots for mpg vs. (wt + qsec) * factor(am)}\label{fig:unnamed-chunk-34}
\end{figure}


\end{document}
